\chapter{Conclusion}
\label{ch:conclusion}
The whole aim of our project has been the design and the implementation of an hybrid music recommender in order to mitigate the cold-start problem in content-based recommender systems. We investigated several types of hybridisation in recommender systems to choose a suitable architecture (shown in~\ref{fig:generalhybrid}) for the available datasets. To represent real world users and raw waveforms, we decided to investigate and implement state-of-the-art techniques.

Despite of the success in computer vision field, we found in our project that convolutional deep neural networks achieve similar results to long-established music genre classifier approaches in music information retrieval field.

Due to the natural selection concept associated to estimation of distribution algorithms, we investigated and considered these optimisation techniques for modelling users' listening behaviour in terms of probabilities of music genres from the songs in they listened.

On the other hand, we found that a limited number of genres for song representation lead us to coarse predictions according to decision-based metrics.

\section{Future work}

For the future, we have the intention to enhance our hybrid music recommender considering a wide range of music genres or latent vectors for item representation. We shall work on investigating several configurations of convolutional deep neural networks and different types of deep learning techniques, particularly, unsupervised learning approaches, for a better high-level representation of audio waveforms. In addition, we will continue investigating the fascinating estimation of distribution algorithms, considering another fitness functions to optimise, to model user profiles in recommender systems. Finally, we also consider the evaluation of hybrid recommender with an online experiment.